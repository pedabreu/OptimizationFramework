\documentclass[a4paper,10pt]{article}
\usepackage[utf8x]{inputenc}

%opening
\title{}
\author{}

\begin{document}

\maketitle

\begin{abstract}

\end{abstract}

\section{25 de Junho de 2012}

\subsection{Smart Swap Genetic Component}

- Ideia de fazer experiencias com instancias 4x4

- Hoje fiz os estudos com as features do grupo "New Sequence Jobs Distribution for Instances 3x3" com excelentes resultados 0.28 de RMSE com treino de 0.8 do dataset (35000)  

- Acrescentei um novo grupo "New Sequence Machines Distribution for Instances 3x3". No fundo é semelhante ao "New Sequence Jobs Distribution for Instances 3x3" mas para as máquinas

- Tive a ideia de partir a nova sequencia formada pelo swap em várias e descrever cada uma delas. Vou usar o método addPartition e criar estas features apartir dos grupos "New Sequence Jobs Distribution for Instances 3x3" e "New Sequence Machines Distribution for Instances 3x3". Assim fico para cada job e cada machine a quantidade que aparece em cada partição da sequencia


\subsection{Outros}
- Tive a ideia de fazer estudos das diferenças de fitness para instancias c a mesma duração (para o PHD)

\section{26 de Junho de 2012}

\subsection{Smart Swap Genetic Component}

-Ideia: Criar um novo target que meça a diferença de horários vistos que é o que uma mutação deve fazer. Uma proposta é a media da difença dos start times a dividir pelos total dos gaps

Reunião com o Carlos
- Mostrar os resultados comparados do Sandeep e do grupo "New Sequence Jobs Distribution for Instances 3x3". Mencionar que as features de index criadas tem desempenhos semelhantes ao Sandeep

- Testar o do grupo "New Sequence Jobs Distribution for Instances 3x3" com 4x4 ou maior, pois estes descrevem totalmente a nova sequencia apesar de nao terem sido feitos especificamente para isso (min,max e mediana)

-Dizer que tanto para dataset de treino pequenos e grandes os resultados são interessantes apesar de piores. O que pode ser utilizado numa corrida do mm algoritmo

- Continuar a criar features que meçam a nova sequencia.

\subsection{Corenet}

- Vi o Pentaho 4.5  Fora a criaçao dos data sources de forma automatica para DW em estrela, nao parece haver novidades. 
- Vi uma demo do Saiku 2.3 e não tinha percentagem



\section{27 de Junho de 2012}

- Explorar o MySQL CLuster para nao ficar dependente do Rank.fep.up.pt. Para testar instalei o VirtualBox
- Criei tabela e codigo para as experiencias Swap Sequence de tamanho 5x5



\end{document}
